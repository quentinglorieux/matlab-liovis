\documentclass[11pt,a4paper]{article}
\usepackage[utf8]{inputenc}
\usepackage[T1]{fontenc}
\usepackage[french]{babel}
\usepackage{amsmath,amssymb}
\usepackage{geometry}
\usepackage{enumitem}
\usepackage{lmodern}

\geometry{margin=1.5cm}

\begin{document}

\begin{center}
    \LARGE QCM --- Cours 1 : Introduction à Matlab\\[-1em]
\end{center}

\bigskip

\begin{enumerate}[label=\textbf{\arabic*.}]

\item Quelle est la structure de base d'un script Matlab ?
\begin{enumerate}[label=\alph*)]
    \item Une fonction avec un retour obligatoire
    \item Un fichier \texttt{.m} exécuté de haut en bas
    \item Un fichier compilé avant exécution
    \item Un fichier indépendant du Workspace
\end{enumerate}

\item Laquelle des propositions décrit correctement une \textbf{fonction} Matlab ?
\begin{enumerate}[label=\alph*)]
    \item Elle modifie toujours le Workspace
    \item Elle possède son propre espace de variables
    \item Elle ne peut pas avoir de sortie
    \item Elle ne peut être appelée que dans un script
\end{enumerate}

\item Dans Matlab, l'indexation commence :
\begin{enumerate}[label=\alph*)]
    \item à 0
    \item à 1
    \item à $-1$
    \item cela dépend du type de variable
\end{enumerate}

\item La commande \texttt{linspace(0,10,100)} génère :
\begin{enumerate}[label=\alph*)]
    \item un vecteur de 100 valeurs espacées d'un pas de 1
    \item un vecteur de 100 valeurs entre 0 et 10
    \item un vecteur de taille 101
    \item une matrice $10\times 100$
\end{enumerate}

\item Que fait la commande \texttt{zeros(3,2)} ?
\begin{enumerate}[label=\alph*)]
    \item Crée un vecteur ligne de trois zéros
    \item Crée une matrice $3\times 2$ de zéros
    \item Crée une matrice $2\times 3$ de zéros
    \item Efface trois variables
\end{enumerate}

\item Laquelle de ces instructions effectue une multiplication \textbf{élément par élément} ?
\begin{enumerate}[label=\alph*)]
    \item \texttt{A * B}
    \item \texttt{A .* B}
    \item \texttt{A . B}
    \item \texttt{A x B}
\end{enumerate}

\item La commande \texttt{A'} correspond à :
\begin{enumerate}[label=\alph*)]
    \item l'inverse de A
    \item l'adjointe (transposée conjuguée) de A
    \item la dérivée de A
    \item la symétrisation de A
\end{enumerate}

\item Quel fichier doit contenir la fonction \texttt{carre(x)} ?
\begin{enumerate}[label=\alph*)]
    \item \texttt{script.m}
    \item \texttt{function.m}
    \item \texttt{carre.m}
    \item n'importe quel nom
\end{enumerate}

\item Que fait l'instruction \texttt{plot(x,y)} ?
\begin{enumerate}[label=\alph*)]
    \item Affiche une image
    \item Trace une courbe de $y$ en fonction de $x$
    \item Calcule la FFT de $y$
    \item Enregistre une figure
\end{enumerate}

\item Que permet \texttt{hold on} ?
\begin{enumerate}[label=\alph*)]
    \item Remplacer la figure existante
    \item Ajouter des courbes sur la même figure
    \item Effacer la figure
    \item Changer l'échelle des axes
\end{enumerate}

\item La commande \texttt{subplot(2,1,1)} prépare une figure :
\begin{enumerate}[label=\alph*)]
    \item composée de 2 lignes et 1 colonne, zone 1
    \item de 1 ligne et 2 colonnes, zone 1
    \item de 2 graphiques superposés non indépendants
    \item de 2 fenêtres séparées
\end{enumerate}

\item \texttt{imagesc(I)} :
\begin{enumerate}[label=\alph*)]
    \item efface la matrice
    \item affiche la matrice comme image
    \item convertit la matrice en double
    \item calcule le spectre de l'image
\end{enumerate}

\item Quel est l'intérêt principal d'une fonction ?
\begin{enumerate}[label=\alph*)]
    \item Augmenter la taille du Workspace
    \item Réutiliser et structurer du code
    \item Accéder directement aux variables globales
    \item Modifier automatiquement le dossier courant
\end{enumerate}

\item Quelle commande active la grille sur une figure ?
\begin{enumerate}[label=\alph*)]
    \item \texttt{grid on}
    \item \texttt{meshgrid}
    \item \texttt{show grid}
    \item \texttt{enableGrid()}
\end{enumerate}

\item Quelle commande permet d'ajouter un titre ?
\begin{enumerate}[label=\alph*)]
    \item \texttt{figtitle}
    \item \texttt{title()}
    \item \texttt{label()}
    \item \texttt{text()}
\end{enumerate}

\item Une variable définie dans un script :
\begin{enumerate}[label=\alph*)]
    \item disparaît immédiatement
    \item reste dans le Workspace
    \item n'est accessible que dans les fonctions
    \item devient automatiquement globale
\end{enumerate}

\item Syntaxe correcte d'une fonction à deux sorties :
\begin{enumerate}[label=\alph*)]
    \item \texttt{function sortie1, sortie2 = f(x)}
    \item \texttt{function [s1, s2] = f(x)}
    \item \texttt{function f(x) = [s1, s2]}
    \item \texttt{function \{s1, s2\} = f(x)}
\end{enumerate}

\item Quel vecteur a un \textbf{pas fixe} ?
\begin{enumerate}[label=\alph*)]
    \item \texttt{linspace(0,1,1000)}
    \item \texttt{0:0.001:1}
    \item \texttt{time(0,1,1000)}
    \item \texttt{seq(0,1,0.001)}
\end{enumerate}



\bigskip

\begin{center}
\Large \textbf{Grille de réponses}
\end{center}

\bigskip

\begin{tabular}{|c|c|c|c|c|c|c|c|c|c|c|c|c|c|c|c|c|c|c|}
\hline
\textbf{Questions} & 1 & 2 & 3 & 4 & 5 & 6 & 7 & 8 & 9 & 10 & 11 & 12 & 13 & 14 & 15 & 16 & 17 & 18 \\
\hline
\textbf{Réponses} & & & & & & & & & & & & & & & & & & \\
\hline
\textbf{Correction} & & & & & & & & & & & & & & & & & & \\
\hline
\end{tabular}

\end{enumerate}

%\newpage
%\section*{Corrigé (enseignant)}
%% Indiquez ici le barème et les réponses si vous le souhaitez.

\end{document}